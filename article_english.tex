\documentclass[12pt,]{article}
\usepackage{lmodern}
\usepackage{amssymb,amsmath}
\usepackage{ifxetex,ifluatex}
\usepackage{fixltx2e} % provides \textsubscript
\ifnum 0\ifxetex 1\fi\ifluatex 1\fi=0 % if pdftex
  \usepackage[T1]{fontenc}
  \usepackage[utf8]{inputenc}
\else % if luatex or xelatex
  \ifxetex
    \usepackage{mathspec}
  \else
    \usepackage{fontspec}
  \fi
  \defaultfontfeatures{Ligatures=TeX,Scale=MatchLowercase}
\fi
% use upquote if available, for straight quotes in verbatim environments
\IfFileExists{upquote.sty}{\usepackage{upquote}}{}
% use microtype if available
\IfFileExists{microtype.sty}{%
\usepackage{microtype}
\UseMicrotypeSet[protrusion]{basicmath} % disable protrusion for tt fonts
}{}
\usepackage[margin=1in]{geometry}
\usepackage{hyperref}
\hypersetup{unicode=true,
            pdftitle={Nice guys or potential cooperators when keeping promises: an experimental and Bayesian account for two explanations},
            pdfauthor={Said Jiménez},
            pdfborder={0 0 0},
            breaklinks=true}
\urlstyle{same}  % don't use monospace font for urls
\usepackage{graphicx,grffile}
\makeatletter
\def\maxwidth{\ifdim\Gin@nat@width>\linewidth\linewidth\else\Gin@nat@width\fi}
\def\maxheight{\ifdim\Gin@nat@height>\textheight\textheight\else\Gin@nat@height\fi}
\makeatother
% Scale images if necessary, so that they will not overflow the page
% margins by default, and it is still possible to overwrite the defaults
% using explicit options in \includegraphics[width, height, ...]{}
\setkeys{Gin}{width=\maxwidth,height=\maxheight,keepaspectratio}
\IfFileExists{parskip.sty}{%
\usepackage{parskip}
}{% else
\setlength{\parindent}{0pt}
\setlength{\parskip}{6pt plus 2pt minus 1pt}
}
\setlength{\emergencystretch}{3em}  % prevent overfull lines
\providecommand{\tightlist}{%
  \setlength{\itemsep}{0pt}\setlength{\parskip}{0pt}}
\setcounter{secnumdepth}{5}
% Redefines (sub)paragraphs to behave more like sections
\ifx\paragraph\undefined\else
\let\oldparagraph\paragraph
\renewcommand{\paragraph}[1]{\oldparagraph{#1}\mbox{}}
\fi
\ifx\subparagraph\undefined\else
\let\oldsubparagraph\subparagraph
\renewcommand{\subparagraph}[1]{\oldsubparagraph{#1}\mbox{}}
\fi

%%% Use protect on footnotes to avoid problems with footnotes in titles
\let\rmarkdownfootnote\footnote%
\def\footnote{\protect\rmarkdownfootnote}

%%% Change title format to be more compact
\usepackage{titling}

% Create subtitle command for use in maketitle
\providecommand{\subtitle}[1]{
  \posttitle{
    \begin{center}\large#1\end{center}
    }
}

\setlength{\droptitle}{-2em}

  \title{Nice guys or potential cooperators when keeping promises: an
experimental and Bayesian account for two explanations}
    \pretitle{\vspace{\droptitle}\centering\huge}
  \posttitle{\par}
    \author{Said Jiménez}
    \preauthor{\centering\large\emph}
  \postauthor{\par}
      \predate{\centering\large\emph}
  \postdate{\par}
    \date{July 2019}

\usepackage{booktabs}
\usepackage{longtable}
\usepackage{array}
\usepackage{multirow}
\usepackage{wrapfig}
\usepackage{float}
\usepackage{colortbl}
\usepackage{pdflscape}
\usepackage{tabu}
\usepackage{threeparttable}
\usepackage{threeparttablex}
\usepackage[normalem]{ulem}
\usepackage{makecell}
\usepackage{xcolor}

\begin{document}
\maketitle

\hypertarget{introduction}{%
\section{Introduction}\label{introduction}}

Human beings are social animals and we get advantages from being so.
Historically, belonging to a social group has provided protection and
guaranteed access to resources such as food, currently there has been
evidence that belonging to a social circle has a positive effect on
longevity, physical and mental health (Holt-Lunstad, 2018; Tough,
Siegrist, \& Fekete, 2017). The interrelation between social closeness
and cooperation is key to the formation of large groups of individuals
without a genetic relationship, such groups form the basis of
communities, societies, and nations, as well as probably constitute one
of the most fundamental conditions for human survival (Fehr \&
Fischbacher, 2003; Fehr \& Schurtenberger, 2018).

An important requirement for cooperation between individuals is
communication, a meta-analysis of 45 studies reported a large positive
effect (\(d = 1.01\)) of communication in cooperation regardless of the
communication medium (Balliet, 2010). In parallel, it is known that
after communication sessions between completely unknown individuals
there is an increase in subjective social closeness (Aron, Melinat,
Aron, Vallone, \& Bator, 1997). Thus, the findings suggest a positive
relationship between communication, cooperation, and social closeness.

One of the forms of communication that have received the most attention
in psychological game theory is the promise, in an interaction between
sender and recipient, it is believed that promises influence beliefs of
the recipient, generating trust and cooperation (Charness \& Dufwenberg,
2006; Vanberg, 2008). However, there are situations where trust is
betrayed, promises are broken or people deceive, for example,
approximately 11\% of people surveyed by the General Social Survey (GSS)
in 2018 responded that they have had sex with someone different to their
partner while they are married (Smith, Davern, Freese, \& Hout, 2018).
Another example, according to the Washington Post, President Trump has
said 10,796 false or misleading statements in 869 days in office
(Kessler, Rizzo, \& Kelly, 2019).

In the laboratory, these transgressions to trust have been explicitly
studied in experiments where subjects have incentives to lie or not keep
their promises. However, the vast majority of these studies in
economics, psychology, and neurosciences have been conducted in people
who do NOT know each other (to name a few, Gneezy, 2005; Baumgartner,
Fischbacher, Feierabend, Lutz, \& Fehr, 2009; Baumgartner, Gianotti, \&
Knoch, 2013; Charness \& Dufwenberg, 2006; Fischbacher \& Föllmi-Heusi,
2013; Gneezy, Rockenbach, \& Serra-Garcia, 2013; Mazar, Amir, \& Ariely,
2008). In social psychology some studies address deception in close
interpersonal relationships, however, they use self-report measurements
(DePaulo, Ansfield, Kirkendol, \& Boden, 2004; DePaulo \& Kashy, 1998)
or rather explore the development of deception detection skills in
same-sex friends (Anderson, DePaulo, \& Ansfield, 2002). Despite the
widespread presence of interactions between human beings belonging to
the same group, it is not known how social closeness between subjects
can affect keeping or breaking promises.

In this manuscript, we explore the effects of three partners with
different levels of social closeness on keeping or breaking promises, as
well as on cooperation in a trust game. The above, we analyze it in
light of the two main motivations that have been pointed out in the
literature regarding keeping promises (Baumgartner et al., 2009):

\begin{enumerate}
\def\labelenumi{\arabic{enumi}.}
\item
  The \emph{instrumental} suggests that promises are kept to make future
  cooperation easier.
\item
  The \emph{intrinsic} mentions that the promises are kept to do what is
  morally right.
\end{enumerate}

In our study, subjects perform a standard trust game in pairs with three
phases: \emph{first}, the trustee makes a promise to pay half of his
earnings regardless of who his investor is; \emph{second}, the investor
receives the promise and decides if he invests his initial budget;
\emph{third}, in case the investor has given the budget, the trustee
faces the decision to pay or not to pay half of his earnings. To
evaluate the effect of social closeness, our subjects participate in the
role of trustee in front of three partners with different levels of
social closeness (zero, low and high) in the role of the investor: a
\emph{computer}, a \emph{stranger}, and a \emph{friend}.

The manipulations mentioned allow us to evaluate several hypotheses, the
first one was proposed a priori and is derived from a larger research
project (you can check the pre-registration of the hypothesis here:
\url{https://osf.io/u97fd}), while the following are exploratory:

\begin{enumerate}
\def\labelenumi{\arabic{enumi}.}
\item
  Social closeness will reduce the decision of breaking the promise.
  According to the \emph{instrumental} motivation, we expect that
  subjects keep their promises with friends for the purpose of
  facilitating future cooperation, which can be extended even beyond the
  trials in the experiment. We also expect, although to a lesser extent,
  that the subjects keep promises to strangers, with the purpose of
  facilitating cooperation at least during the trials during the
  experiment. Finally, we anticipate that the participants break the
  promises to the computer because it is a partner without social
  closeness and they could not ensure cooperation in future trials. It
  should be noted that if the participants keep the promises to the
  computer, evidence would be given in favor of intrinsic motivation.
\item
  There will be an effect of social closeness on cooperation, regardless
  of promises. We expect that there is more probability of paying the
  \emph{friend} than the \emph{stranger} and more probability of paying
  the stranger than the \emph{computer}. According to the
  \emph{instrumental} motivation, the cooperation will be greater for
  partners with whom the subject anticipates greater cooperation in the
  future (friend \textgreater{} stranger \textgreater{} computer).
\item
  Finally, we will obtain two subsamples according to the cooperation
  rate of the subjects as was done in another study (Baumgartner et al.,
  2009). However, what in the aforementioned study was classified as a
  group of \emph{honest} and \emph{dishonest}, we will show that in our
  sample it is not supported because if the groups differ in their
  payment rate it will be due to the difference in their level of
  commitment expressed by promises.
\end{enumerate}

\hypertarget{methods}{%
\section{Methods}\label{methods}}

\hypertarget{subjects}{%
\subsection{Subjects}\label{subjects}}

We included 45 subjects (15 men), recruited from the National Autonomous
University of Mexico, the age range from 19 to 33 years and their
minimum educational level were bachelor's degrees. Subjects went to the
study with a friend considered \emph{close} by themselves, who fulfilled
the following characteristics: he was matched by sex, did not have a
family bond and was not a person with a sentimental or sexual
relationship. 30 subjects (15 men) performed the task within a Magnetic
Resonance Imaging (MRI) scanner, however, their image data is not
analyzed in the present work.

\hypertarget{task}{%
\subsection{Task}\label{task}}

All subjects performed an adaptation of the trust game with promises,
using hypothetical monetary rewards (Baumgartner et al., 2009). The task
was programmed in PsychoPy2 version 1.84.2 (Peirce et al., 2019; Peirce,
2008) and consisted of 24 trials between two players: trustee and
investor. The trustee originally has 0 Mexican pesos and the investor
has 2 pesos, the investor is presented with the opportunity to give his
money to the trustee or keep it. If he invests his money it is
multiplied by 5 pesos, so that the trustee has 10 pesos. Finally, the
trustee decides to pay half to the investor or keep the 10 pesos.

The structure described is repeated in 24 trials, however, in 4 of the
trials the trustee can send a promise to the investor, the promises were
that \emph{always}, \emph{mostly}, \emph{sometimes}, or \emph{never}
would pay back. Each promise was valid for three trials so that 12 of
the trials have the effect of the promise and the other 12 do not. It is
worth mentioning that the promises are not made to a particular partner,
but the subject expresses their commitment level to pay back regardless
of who their partners are in the next three trials in which the promise
is valid.

The variation with respect to the original task is that in our
experiment three partners were presented with three levels of social
closeness in the role of the investor: \emph{computer} (no closeness),
\emph{strange} (low closeness) and \emph{friend} (high closeness), while
the subject acts as the trustee. Each partner performs the role of
investor in 8 of 24 total trials, however, both the promises and social
closeness conditions were presented to the trustee in pseudorandom
order.

Since our main interest was the trustee's behavior, investors' decisions
were programmed \emph{a priori} to give their amount in 6 trials and in
2 did not. The covert story for all our subjects was that they would be
playing in real-time with their friend, the stranger (was told he would
be another same-sex unknown person) and the computer. A diagram of the
chronology of the experimental task with the duration of each phase in
seconds is shown in Figure 1, each box from left to right represents a
screen that was shown to the subjects sequentially. Section A\(_1\)
corresponds to an example of trials without promises and section A\(_2\)
to trials with promises.

Fixation phase consisted only of a period in which the subject paid
attention without performing any particular behavior, then,
\textbf{promises phase} in part A\(_1\) indicated to the subject that
the following three trials they could decide without the effect of the
promise, while in A\(_2\) the subject was asked to decide between
\emph{always}, \emph{mostly}, \emph{sometimes} or \emph{never} pay back.
This was followed by another fixation period, subsequently, trails in
both A\(_1\) and A\(_2\) continue in the same way.

In the \textbf{anticipation/assignation phase}, subjects were told who
their partner was for that trial (computer, stranger or friend) and was
given the message that their partner was making their decision. Later in
the \textbf{decision phase}, the subject was informed if his partner
invested his \$2 or not, he was reminded of his promise level (if they
were trails with promises) and, in case of his partner had invested, he
was asked to decide whether to pay back or not. Finally, in
\textbf{feedback phase} payments for that trial were shown and the
sequence was repeated.

\begin{figure}

{\centering \includegraphics[width=0.8\linewidth]{/Users/saidjimenez/Documents/R/github_Said/social_closeness/Manuscript/figures/task} 

}

\caption{Trust game with promises}\label{fig:figA}
\end{figure}

\hypertarget{procedure}{%
\subsection{Procedure}\label{procedure}}

Subjects came with a same-sex friend considered close by him or her, it
was emphasized that they must not have a romantic or family relationship
with their friends, to try to exclude the effect on cooperation due to a
consanguineous relationship or sexual attraction. Also, to ensure that
the subjects and their companions had a similar degree of social
closeness to each other, in the laboratory both responded to the
``Inclusion of the Other in Self'' IOS scale (Aron et al., 1997) without
observing their partner responses. The scale consists of seven pairs of
circles that vary in the degree of overlap between them, the respondent
must select the pair of circles that best represents the subjective
closeness to his partner.

Subjects were trained in the task by two of the researchers, the game
was first explained verbally, and then the computerized interface was
presented to them on a portable computer. They were told that they would
make their decisions with four buttons on the computer keyboard, which
corresponded to the levels of promises and with only two of these to
make their decision to pay back or not. Friends heard the explanation,
they were told that they would make their decisions in a separate room
on another computer. Subjects performed between 3 and 6 practice tests
together with the researcher who exemplified the course of the game when
the investor gave his budget. In the practice trials, all the phases
indicated in Figure 1 were shown, however, the identity of the investors
was replaced by question marks.

When the subjects did not report doubts we proceeded to take them
without their companions to another room where they would make their
decisions on a laptop, or, with the Lumina Cedrus response system in the
case that it was a subject to which we also performed MRI. The subjects
began the task believing that they would really play with the three
partners of different social closeness. In the other room, another of
the researchers performed the \emph{debriefing} to the companion, they
were told that the purpose of the research was to know if their mere
presence had an effect on the subject's decisions, but that to simplify
the analysis their responses were programmed \emph{a priori} so your
participation ended there.

When the subject finished the task, \emph{debriefing} was also carried
out, he was told that the decisions of his partners had been programmed.
Although no participant exercised their right, both subjects and their
friends were told that they could withdraw their participation and
informed consent if they considered disagreeing with any of the
manipulations made by researchers.

\hypertarget{results}{%
\section{Results}\label{results}}

\hypertarget{descriptives}{%
\subsection{Descriptives}\label{descriptives}}

We analyzed 810 decisions about paying back or not (18 per subject), as
well as 180 promises (4 per subject), 69\% of the decisions, were pay
back, while the proportion of the promises selected were 5.4\%, 16.2\%,
43.8\% and 40.2\%, respectively for promises \emph{never},
\emph{sometimes}, \emph{mostly} and \emph{always}. Figure 2 shows the
payment proportion depending on the levels of promises and partners.

Breaking promises, that is, subject had promised \emph{always} and
subsequently decides not to pay back occurred only 12.4\% of trials (19
times they did not pay back of 153 when promised \emph{always}), of
which, 8.50\% (13/153) were trials with the computer, 3.92\% (6/153)
with the stranger and 0\% (0/153) with the friend. It is essential to
note that breaking the promise was not the most frequent decision, in
91.5\% of trials when they promised \emph{always} and had the computer
as a partner they decided to pay back.

\begin{figure}

{\centering \includegraphics[width=0.8\linewidth]{article_english_files/figure-latex/fig2-1} 

}

\caption{Pay back proportion by partners and promises levels}\label{fig:fig2}
\end{figure}

\hypertarget{hierarchical-bayesian-modeling}{%
\subsection{Hierarchical Bayesian
Modeling}\label{hierarchical-bayesian-modeling}}

All models presented below were programmed in R via the \textbf{brms}
package (Bürkner, 2017; R Core Team, 2019), which performs the inference
using sampling by Markov chain Monte Carlo through \textbf{Stan} (Stan
Development Team, 2018). For each model, the posteriors distributions of
all parameters were approximated with four chains of 2000 iterations
each, the first 1000 iterations of each chain were discarded (burning
period), for a total of 4000 post-burning samples. Models convergence
was evaluated through visual inspection of the chains and calculation of
the \(\hat{R}\) statistic, which for all parameters was 1, which is
interpreted as convergence. Vaguely informative prior distributions were
used for the parameters of interest in the models, which allows the data
to dominate the inference, also assumes as little as possible regarding
the nature of the phenomenon, which could be adequate for the current
state of evidence in the problem we are studying (McElreath, 2018).

\hypertarget{social-closeness-between-partners}{%
\subsubsection{Social closeness between
partners}\label{social-closeness-between-partners}}

To assess social closeness, 30 (15 men) of our subjects and their
friends responded to the IOS scale (Aron et al., 1997; Fareri \&
Delgado, 2014; Sip, Smith, Porcelli, Kar, \& Delgado, 2015), which
consist of seven pairs of circles that vary in the degree of overlap and
represent the social closeness that an individual perceives with respect
to the other. A Hierarchical and Cumulative Bayesian Model (See section:
Commitment expressed in the promises), supports the hypothesis that
there are no differences in subjective social closeness between subjects
and their friends, with a posterior evidence ratio of 2.74 in its favor.

\hypertarget{effect-of-social-closeness-on-breaking-promises}{%
\subsubsection{Effect of social closeness on breaking
promises}\label{effect-of-social-closeness-on-breaking-promises}}

To evaluate the hypothesis that social closeness would reduce the
behavior of breaking the promise, we filtered all trials in which
subjects promised that they would \emph{always} pay back and then
decided not to pay. Subsequently, Bayesian inference was used to assess
the effect of partners on the decision to break the promise at the
individual and population level. For this purpose, a Hierarchical
Bayesian Model was carried out that assumes that the uncertainty in
partners effect on the decision to break the promise varies depending on
each individual, however, it also assumes that these variations belong
to common population distributions (Gelman \& Hill, 2006).

\[
\begin{aligned}
&y_i \sim \mathrm{Bernoulli}(\theta_i) \\
&\mathrm{logit(\theta_i)} = \mathbf{X}\beta  +  \mathbf{Z}u
\end{aligned}
\]

In the previous model, the decision to break the promise \(y_i\) comes
from a Bernoulli distribution with probability \(\theta_i\), the goal of
the Hierarchical Model is to predict each decision through the linear
combination of the effects of each partner, transformed by its inverse
link function \(\mathrm{logit}\) (Bürkner, 2017). In this model,
\(\beta\) and \(u\) are coefficients at the population level and
individual level respectively, while \(\mathbf{X}\), \(\mathbf{Z}\) are
their corresponding design matrices. In this case, the population
coefficients correspond to the presence of partners with no social
closeness \(\beta^{computer}\), low \(\beta^{stranger}\) and high
\(\beta^{friend}\).

Figure 3 shows the posterior probability of breaking the promise
depending on the partner, circles correspond to the medians of the
distributions of the posterior estimates effects, the thick bar, and the
thin bar correspond to the interval of 50\% and 95\% posterior
probability, respectively. It is clear that the probability of breaking
the promise decreases as social closeness increases, however, we
calculate the reasons for evidence for the following hypotheses:

\begin{itemize}
\item
  There is a greater probability to break the promise to the computer
  than the stranger.
\item
  There is more probability to break the promise to the stranger
  compared to a friend.
\end{itemize}

The evidence ratio, which is the ratio between the posterior probability
of the mentioned hypotheses and their corresponding alternative
hypotheses was 7.47 and 8.64 respectively in favor of the previous
hypotheses. As can be seen in Figure 5 in the estimate for
\(\beta^{computer}\), when there is no social closeness we have a 95\%
posterior probability that the parameter for promise-breaking is between
4-38\%, although it is a small proportion, it clearly excludes the
probability that promises are not broken.

\begin{figure}

{\centering \includegraphics[width=0.8\linewidth]{article_english_files/figure-latex/fig3-1} 

}

\caption{Posterior probability of Promise Breaking}\label{fig:fig3}
\end{figure}

\hypertarget{effects-of-promises-and-social-closeness-on-cooperation}{%
\subsubsection{Effects of promises and social closeness on
cooperation}\label{effects-of-promises-and-social-closeness-on-cooperation}}

To evaluate the effect of the experimental conditions on the decision to
pay back, we used all trials were partners invest. Again, a Hierarchical
Bayesian Model was fit, which assumes that the promises and partners
have an effect that varies for each individual, however, it also assumes
that these variations belong to common population distributions. The
model estimates that the probability \(\theta_i\) of the decision to pay
back is based on the effect of the presence of promises
\(\beta^{promise}\), as well as partners with low \(\beta^{stranger}\)
and high \(\beta^{friend}\) social closeness.

Figure 4 shows the posterior distributions of the \(\beta\) coefficients
in the logit scale, the center line represents the median of the
distribution and the shaded area corresponds to the interval of 50\%
posterior probability. As can be seen, more than 95\% of the posterior
density of the population coefficients is greater than 0, which
indicates strong evidence of the effect of experimental conditions on
the decision to pay back, so that the presence of promises and the
stranger increase clearly the probability of paying, but the presence of
the friend is the condition that has the greatest effect on this
behavior. Posterior distributions clearly show that the probability of
cooperation increases as social closeness does.

Also, Figure 5 presents the posterior predictive distribution in
contrast to the proportion of decisions to each partner during the
promises phase, each panel corresponds to one of the first twelve
subjects. The responses on the dotted line would indicate that the
subject paid back random to that partner. Posterior predictive
distribution simulates observations of the model and compares them with
the actual data, it helps us to identify if the model is sufficiently
close to the process that generated the data (Lee \& Wagenmakers, 2014;
Schad, Betancourt, \& Vasishth, 2019). It can be seen that there is a
great correspondence between the subject's responses and predictions of
the model, even in cases where the model is ``wrong'' (for example,
subject 1), the observed response is in the range of a predicted
standard deviation, which gives credibility to estimates.

\begin{figure}

{\centering \includegraphics[width=0.8\linewidth]{article_english_files/figure-latex/fig4-1} 

}

\caption{Posterior estimates in logit scale}\label{fig:fig4}
\end{figure}

\begin{figure}

{\centering \includegraphics[width=1\linewidth,height=1\textheight]{article_english_files/figure-latex/fig5-1} 

}

\caption{Posterior predictive over pay back rate}\label{fig:fig5}
\end{figure}

\hypertarget{effects-of-social-closeness-on-cooperation-varying-by-promises}{%
\subsubsection{Effects of social closeness on cooperation varying by
promises}\label{effects-of-social-closeness-on-cooperation-varying-by-promises}}

In this model, it is assumed that social closeness has an effect that
varies for each level of promise, which implies that there are levels of
promise that are more sensitive than others to the effect of the
partners on the decision to pay back. Again, a Hierarchical Bayesian
Model was made to estimate the effect of partners at the population
level and the variations depending on the promise levels. Table 1
summarizes the subsequent distributions of the model coefficients in the
logit scale, including point estimates, standard errors and Bayesian
Credibility Intervals of 95\%. An estimate similar to the previous
models can be observed, with strong evidence of the effect of social
closeness on the decision to pay back. Although the credibility range
for the friend's effect includes 0, the evidence ratio that the effect
is greater than zero is 15.81 with 94\% of posterior probability.

On the other hand, Figure 6 shows the posterior predictive distribution,
compared to the payment rates of all individuals to the different
partners and their variation by the level of promise. With the exception
of the promise \emph{never}, a monotonic positive effect of partners is
observed at all levels of promise, however, it can be observed how the
effect of social closeness varies depending on the strength of the
promise, mainly for the decisions towards the computer.

\begin{figure}

{\centering \includegraphics[width=0.8\linewidth]{article_english_files/figure-latex/fig6-1} 

}

\caption{Posterior predictive over pay back rate, Social Closeness Varying effects by Promise levels}\label{fig:fig6}
\end{figure}

\begin{table}[!h]

\caption{\label{tab:tabla}Posterior coefficients estimates}
\centering
\begin{tabular}{lrrrr}
\toprule
\multicolumn{1}{c}{ } & \multicolumn{2}{c}{ } & \multicolumn{2}{c}{95 \% CI} \\
\cmidrule(l{3pt}r{3pt}){4-5}
Term & Estimate & Est.Error & Lower & Upper\\
\midrule
Intercept & -0.153 & 0.536 & -1.005 & 0.617\\
Stranger & 0.829 & 0.410 & 0.128 & 1.470\\
Friend & 2.084 & 1.248 & -0.151 & 3.975\\
\bottomrule
\end{tabular}
\end{table}

\hypertarget{cumulative-bayesian-modeling}{%
\subsection{Cumulative Bayesian
Modeling}\label{cumulative-bayesian-modeling}}

\hypertarget{commitment-expressed-in-promises}{%
\subsubsection{Commitment expressed in
promises}\label{commitment-expressed-in-promises}}

In the original study of promises, authors chose to divide their sample
into two according to the hierarchical clustering technique with Ward's
method (Baumgartner et al., 2009). In this way they obtained two sets of
participants that were different in their payment rates, despite the
fact that they both made very high promises, for this reason, the
authors named the group that paid little \emph{dishonest} and the group
that paid a lot \emph{honest}. In a similar exercise, in the present
study we performed the hierarchical grouping technique to obtain a
solution of 2 groups and we found two similar sets in \(n\) that we call
the group ``Low'' and ``High'' (Low = 25, High = 20), with quite
different payment ratios (Low = 58\%, High = 83\%), and a 95\%
confidence interval of 20\% to 32\% in the difference favor of the High
group.

Although it seems a similar result to that reported in that paper, we
explore the pattern of promises of both groups to determine if it was
possible to classify our subjects as \emph{honest} and \emph{dishonest}.
If we hypothesize that the commitment to pay would be reflected in the
level of promises selected, a group of \emph{dishonest} people could
generate in their partners the belief that they will pay by choosing a
high level of promises (mostly or always) and, later, betraying that
trust when deciding not to pay back.

In order to explore whether the groups represent populations that do not
differ in the level of commitment expressed in the promises, we use a
Cumulative Bayesian Model, which assumes that the levels of promises are
an observed ordinal variable \(Y\) that originates from the
categorization of a continuous latent variable \(\tilde{Y}\), for this
case, \emph{the expressed commitment to pay back} (Bürkner \& Vuorre,
2019). The degree to which the subjects of the High group differ from
the Low group, in normal standard deviations (\(z\)-values), on the
latent scale of \(\tilde{Y}\), has a point estimate of 0.95, which
implies that the High group has 0.95 \(z\)-values greater commitment to
pay back than the Low group.

The 95\% Bayesian Credibility Interval indicates that the High group is
between 0.40 to 1.51 \(z\)-values of difference from Low. So we can
conclude with at least a 95\% probability that people belonging to the
High group expressed in their promises a greater commitment to pay back
than the subjects of the Low group. If we look at Figure 7, the
probability of choosing the different levels of promises varies
depending on the group, the one who had the highest percentage of
decisions to pay back is more likely to choose \emph{always} (High
group), while the group that had the lowest percentage of decisions to
pay back are more likely to choose \emph{mostly} (Low group).

According to our data, we could not justify the classification of our
groups according to their honesty or dishonesty, at least not with the
technique used in Baumgartner et al. (2009). Since people were quite
consistent with keeping what they promised to pay.

\begin{figure}

{\centering \includegraphics[width=0.8\linewidth]{article_english_files/figure-latex/fig7-1} 

}

\caption{Promise choices by group}\label{fig:fig7}
\end{figure}

\hypertarget{discussion}{%
\section{Discussion}\label{discussion}}

The objective of this study was to evaluate the effect of social
closeness on keeping and breaking promises, as well as on cooperation in
a trust game. Our results give evidence that zero social closeness
increases the probability of breaking the promise and it decreases
monotonically as social closeness increases. Likewise, as social
closeness increases, so does cooperation; \emph{high} social closeness
has an effect on decisions that surpasses even those that all other
experimental conditions. The most interesting finding was that when
subjects express high commitment to cooperate (through choosing that
they would \emph{always} pay back) probability that they comply is also
quite high, however, there is a proportion of transgressions that
according to our inferences are between 4 - 38\% with 95\% certainty.

To our knowledge, this is the first experimental study that incorporates
social closeness as a predictor of the decision to break promises with
socially relevant partners. In previous studies, the participants remain
anonymous during the course of the tasks (Baumgartner et al., 2009),
their measurements to evaluate transgressions are self-reported (DePaulo
et al., 2004; DePaulo \& Kashy, 1998), they do not directly quantify the
breaking of promises (Charness \& Dufwenberg, 2006; Vanberg, 2008), or
as we will discuss later, they use heterogeneous measures of social
closeness (Glaeser, Laibson, Sceinkman, \& Soutter, 2000).

Our experiment is probably the first to include a partner without social
closeness (\emph{computer}), our data give evidence that the mere fact
of considering that subjects play with a human diminishes the
probability of breaking the promise. It is approximately 7.5 times more
likely to break the promise to the computer than to the stranger, even
though that partner was not known. In our study, zero social closeness
is important because it allows us to explore the \emph{intrinsic}
motivation to keep the promises, if humans keep their word because it is
morally correct, we would anticipate that probability of breaking the
promise was very low in the three partners and, particularly, the
inference regarding the probability of breaking the promise to the
computer would include zero.

The probability of breaking the promise without social closeness,
although low, is clearly greater than zero, which provides evidence that
contradicts, at least in degree, the hypothesis of \emph{intrinsic}
motivation. We cannot exclude the possibility that the subjects keep
promises mainly morally motivated, because even in the case of the
computer, the subjects keep a large proportion, but we can exclude that
this is their only motivation.

On the other hand, high social closeness allows us to explore the
\emph{instrumental} motivation to keep promises. In our study, it is
almost 9 times more likely to break the promise to the stranger than to
the friend; if human beings keep their word to facilitate cooperation in
the future, we would anticipate that with a partner of high social
closeness (with whom they are very likely to cooperate, even after the
experiment) there would be less chance of breaking the promise. In fact,
in our study, the promise to friends was not broken on any occasion, so
the inference, in this case, is obvious.

Our findings could be explained by the social norm called
\emph{conditional cooperation}, which indicates that the belief that
other people cooperate at high levels also induces high levels of
cooperation (Fehr \& Schurtenberger, 2018). Thus, the lack of certainty
regarding the decisions that the computer would make could explain the
probability of breaking the promise that exists towards this partner.
Similarly, the information that the subjects have regarding friend's
behavior -even before the experiment- can explain the high levels of
cooperation and keeping of promises towards this partner.

Regarding cooperation, it is not the first investigation in which the
effect of social closeness is evaluated, however, the nature of
experimental manipulation has been diverse. For example, in one study
the contribution in a public goods game was evaluated during several
trials in a group made up of the same individuals (\emph{partners}),
compared to another group of new different subjects for each trial
(\emph{strangers}) (Croson, 1996). It was assumed that in the condition
of \emph{partners} the cooperation would be greater because it is a
stable group compared to the group of \emph{strangers}. The mentioned
study does not give details regarding the recruitment of participants,
so we could assume that even in the condition of \emph{partners} these
are individuals who do not consider themselves socially close.

In another experiment, an increase in cooperation in a trust game in
pairs was reported when individuals are socially close (Glaeser et al.,
2000), in this study, subjects know each other and the researchers
carefully measured several variables regarding their social connection,
however, some individuals who arrived together were allowed to perform
the task between themselves and others were paired with unclear
criteria. The above allows social closeness between partners to be
heterogeneous, allows for romantic relationships and assumes that two
individuals who come together to class are considered close to each
other. Also, since it is a study of \emph{one-shot}, it excludes the
possibility of evaluating how the same subject varies his behavior based
on different levels of social closeness.

To try to homogenize social closeness between our subjects and their
partners, participants performed the task with a same-sex match friend
considered close by themselves, in none case was it a relative and,
although we understand that there may be same-sex couples and that the
subjects may have hidden their relationship, they were emphasized that
to be part of the study their companion could not be a romantic or
sexual partner. Besides, Bayesian analysis of the IOS scale showed
evidence in favor that the degree of social closeness between 30 of our
45 subjects and their friends was the same. Although the results of the
two studies mentioned are practically in the same direction as the
findings of our experiment, establishing the similarity in effects
magnitudes is not possible due to the design differences, the performed
tasks, and the statistical procedures used.

A methodological contribution of our study is that, in the literature of
promises, it is the first to use Bayesian inference tools:
\emph{Hierarchical Models} and \emph{Cumulative Models}. The former (aka
multilevel models), allowed us to model the effect of the experimental
conditions on the individual response, these models assume that lower
level observations (e.g.~decisions of each individual) are nested in
higher level units (e.g.~individual subjects). Within-subjects designs
have traditionally been analyzed with repeated measures AN(C)OVA,
however, \emph{Hierarchical Models} grant the advantages of naturally
dealing with unbalanced data, including categorical and/or numerical
predictors, explicitly incorporate individual variability, among others
(Bürkner, 2017; Gelman \& Hill, 2006; McElreath, 2018; Vuorre \& Bolger,
2018).

In our case, as shown in Figure 5, the \emph{Hierarchical Model} allowed
us to capture very precisely how the experimental conditions affected
the decisions of each subject and the differences between them. For
example, there are obviously motivated \emph{instrumental} subjects such
as number 8 or 12 who are very sensitive to the identity of their
investors and modify their cooperation under social closeness. At the
same time, there are notably \emph{intrinsic} subjects such as 9 and 10
who cooperate all the time regardless of who their partners are. The
\emph{Hierarchical Model} naturally includes this information for the
estimation of population effects, which, if not considered, would lead
to inaccurate inferences.

On the other hand, the \emph{Cumulative Model} allowed us to capture the
strength of the commitment expressed through promises. A very frequent
problem that has been pointed out recently is that analyzing ordinal
data with methods that assume that observations are \emph{metric} can
lead to serious inference errors (Liddell \& Kruschke, 2018). The
\emph{Cumulative Models} assume that the observed responses come from
the categorization of a continuous latent variable (Bürkner \& Vuorre,
2019), so in our study, we could identify that the groups with different
payment proportions, which in other articles have been called
\emph{dishonest} and \emph{honest}, differ in the commitment they
express through their promises. Considering that the tools to perform
the \emph{Cumulative Models} are relatively recent, it would be valuable
to replicate published studies (e.g., Baumgartner et al., 2009, 2013)
considering the new alternatives of inference.

Although the \emph{Hierarchical} and \emph{Cumulative} models are not
tools exclusively for Bayesian inference, their application from this
approach represents several advantages compared to frequentist
statistics. Classic problems of the previous one such as multiple
comparisons, the decision of when to stop collecting subjects (stopping
rule), or use of planned comparisons versus \emph{post hoc}, are not
factors that affect the support provided by the data to the hypothesis
from the Bayesian approach (Dienes, 2011). In Bayesian inference,
hypothesis testing makes formal use of probability to express the
plausibility of theories, in our case we were able to obtain evidence
ratios regarding the extent our data support hypotheses, regardless
these were proposed \emph{a priori} or \emph{post hoc}.

A limitation of this paper is the use of hypothetical rewards compared
to real rewards. A relatively recent study on decision-making reported
that there is more loss aversion when subjects have real monetary
rewards compared to hypothetical in a risk task (Xu et al., 2016).
However, other works do not report differences between the use of both
types of rewards in self-control, temporal and social discount tasks
(Johnson \& Bickel, 2006; Locey, Jones, \& Rachlin, 2011; Rachlin \&
Jones, 2006). Likewise, it can be argued that our results have
theoretical congruence (Fehr \& Schurtenberger, 2018) and are in the
same direction as other works that use real money (Croson, 1996; Glaeser
et al., 2000), so there are not many reasons to expect that other types
of rewards would modify our results.

Another possible limitation is the use of multiple trials with each
partner. Although the decision to use repeated measures during our
design serves more the purpose of reducing contaminant sources and
increasing internal validity (Maxwell, Delaney, \& Ken, 2018), to have
more clarity regarding the difference in the \emph{intrinsic} and
\emph{instrumental} motivations, the studies could benefit from only
include one trial for each partner. Play a \emph{one-shot} with each
partner avoids the possibility that multiple trials could generate the
belief that the computer can also vary its behavior according to the
decisions of the \emph{trustee}.

\hypertarget{conclusions}{%
\section{Conclusions}\label{conclusions}}

Subjects keep their promises by a combination of \emph{intrinsic} and
\emph{instrumental} motivations. They are predominantly \emph{nice guys}
because they keep their word even with investors with no social
closeness. However, its trustworthiness is far from perfect, since there
is a small, but reliably greater than zero, proportion of betrayals
committed towards this partner. It seems that if subjects cannot assure
that their investor will be a \emph{potential cooperator}, dishonesty
may emerge in the form of broken promises. The theoretical predictions
of the social norm \emph{conditional cooperation} frame the finding that
social closeness decreases the probability of breaking the promise and
increases cooperation. In the same way that suggests the correspondence
between commitment expressed through promises and subsequent behavior.

\hypertarget{bibliography}{%
\section*{Bibliography}\label{bibliography}}
\addcontentsline{toc}{section}{Bibliography}

\hypertarget{refs}{}
\leavevmode\hypertarget{ref-Anderson2002}{}%
Anderson, D. E., DePaulo, B. M., \& Ansfield, M. E. (2002). The
development of deception detection skill: A longitudinal study of
same-sex friends. \emph{Personality and Social Psychology Bulletin},
\emph{28}(4), 536--545. \url{https://doi.org/10.1177/0146167202287010}

\leavevmode\hypertarget{ref-Aron1997}{}%
Aron, A., Melinat, E., Aron, E. N., Vallone, R. D., \& Bator, R. J.
(1997). The experimental generation of interpersonal closeness: A
procedure and some preliminary findings. \emph{Personality and Social
Psychology Bulletin}, \emph{23}(4), 363--377.
\url{https://doi.org/10.1177/0146167297234003}

\leavevmode\hypertarget{ref-Balliet2010}{}%
Balliet, D. (2010). Communication and cooperation in social dilemmas: A
meta-analytic review. \emph{Journal of Conflict Resolution},
\emph{54}(1), 39--57. \url{https://doi.org/10.1177/0022002709352443}

\leavevmode\hypertarget{ref-Baumgartner2009}{}%
Baumgartner, T., Fischbacher, U., Feierabend, A., Lutz, K., \& Fehr, E.
(2009). The Neural Circuitry of a Broken Promise. \emph{Neuron},
\emph{64}(5), 756--770.
\url{https://doi.org/10.1016/j.neuron.2009.11.017}

\leavevmode\hypertarget{ref-Baumgartner2013}{}%
Baumgartner, T., Gianotti, L. R. R., \& Knoch, D. (2013). Who is honest
and why: Baseline activation in anterior insula predicts
inter-individual differences in deceptive behavior. \emph{Biological
Psychology}, \emph{94}(1), 192--197.
\url{https://doi.org/10.1016/j.biopsycho.2013.05.018}

\leavevmode\hypertarget{ref-Burkner2017}{}%
Bürkner, P.-C. (2017). brms : An R Package for Bayesian Multilevel
Models Using Stan. \emph{Journal of Statistical Software}, \emph{80}(1).
\url{https://doi.org/10.18637/jss.v080.i01}

\leavevmode\hypertarget{ref-Burkner2019}{}%
Bürkner, P.-C., \& Vuorre, M. (2019). Ordinal Regression Models in
Psychology: A Tutorial. \emph{Advances in Methods and Practices in
Psychological Science}, \emph{2}(1), 77--101.
\url{https://doi.org/10.1177/2515245918823199}

\leavevmode\hypertarget{ref-Charness2006}{}%
Charness, G., \& Dufwenberg, M. (2006). Promises and partnership.
\emph{Econometrica}, \emph{74}(6), 1579--1601.
\url{https://doi.org/10.1111/j.1468-0262.2006.00719.x}

\leavevmode\hypertarget{ref-Croson1996}{}%
Croson, R. T. (1996). Partners and strangers revisited. \emph{Economics
Letters}, \emph{53}(1), 25--32.
\url{https://doi.org/10.1016/S0165-1765(97)82136-2}

\leavevmode\hypertarget{ref-DePaulo2004}{}%
DePaulo, B. M., Ansfield, M. E., Kirkendol, S. E., \& Boden, J. M.
(2004). Serious Lies. \emph{Basic \& Applied Social Psychology},
\emph{26}(2/3), 147--167.
\url{https://doi.org/10.1207/s15324834basp2602\&3_4}

\leavevmode\hypertarget{ref-DePaulo1998}{}%
DePaulo, B. M., \& Kashy, D. A. (1998). Everyday Lies in Close and
Casual Relationships. \emph{Journal of Personality and Social
Psychology}, \emph{74}(1), 63--79.
\url{https://doi.org/10.1037/0022-3514.74.1.63}

\leavevmode\hypertarget{ref-Dienes2011}{}%
Dienes, Z. (2011). Bayesian versus orthodox statistics: Which side are
you on? \emph{Perspectives on Psychological Science}, \emph{6}(3),
274--290. \url{https://doi.org/10.1177/1745691611406920}

\leavevmode\hypertarget{ref-Fareri2014}{}%
Fareri, D. S., \& Delgado, M. R. (2014). Differential reward responses
during competition against in- and out-of-network others. \emph{Social
Cognitive and Affective Neuroscience}, \emph{9}(4), 412--420.
\url{https://doi.org/10.1093/scan/nst006}

\leavevmode\hypertarget{ref-Fehr2003}{}%
Fehr, E., \& Fischbacher, U. (2003). The nature of human altruism.
\emph{Nature}, \emph{425}(6960), 785--791.
\url{https://doi.org/10.1038/nature02043}

\leavevmode\hypertarget{ref-Fehr2018}{}%
Fehr, E., \& Schurtenberger, I. (2018). Normative foundations of human
cooperation review-article. \emph{Nature Human Behaviour}, \emph{2}(7),
458--468. \url{https://doi.org/10.1038/s41562-018-0385-5}

\leavevmode\hypertarget{ref-Fischbacher2013a}{}%
Fischbacher, U., \& Föllmi-Heusi, F. (2013). Lies in disguise-an
experimental study on cheating. \emph{Journal of the European Economic
Association}, \emph{11}(3), 525--547.
\url{https://doi.org/10.1111/jeea.12014}

\leavevmode\hypertarget{ref-Gelman2006}{}%
Gelman, A., \& Hill, J. (2006). \emph{Data analysis using regression and
multilevel/hierarchical models.} (p. 651).
\url{https://doi.org/10.2277/0521867061}

\leavevmode\hypertarget{ref-Glaeser2000}{}%
Glaeser, E., Laibson, D., Sceinkman, J., \& Soutter, C. (2000).
Measuring Trust. \emph{The Quartely Journal of Economics},
\emph{115}(3), 811--846.

\leavevmode\hypertarget{ref-Gneezy2005}{}%
Gneezy, U. (2005). Deception: The role of consequences. \emph{American
Economic Review}, \emph{95}(1), 384--394.
\url{https://doi.org/10.1257/0002828053828662}

\leavevmode\hypertarget{ref-Gneezy2013}{}%
Gneezy, U., Rockenbach, B., \& Serra-Garcia, M. (2013). Measuring lying
aversion. \emph{Journal of Economic Behavior and Organization},
\emph{93}, 293--300. \url{https://doi.org/10.1016/j.jebo.2013.03.025}

\leavevmode\hypertarget{ref-Holt-Lunstad2018}{}%
Holt-Lunstad, J. (2018). Why Social Relationships are Important for
Physical Health: A Systems Approach to Understanding and Modifying Risk
and Protection. \emph{Ssrn}, (October 2017), 1--22.
\url{https://doi.org/10.1146/annurev-psych-122216-011902}

\leavevmode\hypertarget{ref-Johnson2006}{}%
Johnson, M. W., \& Bickel, W. K. (2006). Within-subject comparison of
real and hypothetical money rewards in delay discounting. \emph{Journal
of the Experimental Analysis of Behavior}, \emph{77}(2), 129--146.
\url{https://doi.org/10.1901/jeab.2002.77-129}

\leavevmode\hypertarget{ref-Kessler2019}{}%
Kessler, G., Rizzo, S., \& Kelly, M. (2019). President Trump has made
more than 10,000 false or misleading claims. Retrieved from
\href{https://www.washingtonpost.com/politics/2019/06/10/president-trump-has-made-false-or-misleading-claims-over-days/?utm\%7B/_\%7Dterm=.8565a16aa5e8}{https://www.washingtonpost.com/politics/2019/06/10/president-trump-has-made-false-or-misleading-claims-over-days/?utm\{\textbackslash{}\_\}term=.8565a16aa5e8}

\leavevmode\hypertarget{ref-Lee2014}{}%
Lee, M. D., \& Wagenmakers, E.-J. (2014). \emph{Bayesian Cognitive
Modeling}. \url{https://doi.org/10.1017/cbo9781139087759}

\leavevmode\hypertarget{ref-Liddell2018}{}%
Liddell, T. M., \& Kruschke, J. K. (2018). Analyzing ordinal data with
metric models: What could possibly go wrong? \emph{Journal of
Experimental Social Psychology}, \emph{79}(November 2017), 328--348.
\url{https://doi.org/10.1016/j.jesp.2018.08.009}

\leavevmode\hypertarget{ref-Locey2011}{}%
Locey, M. L., Jones, B. A., \& Rachlin, H. (2011). Real and hypothetical
rewards in self-control and social discounting. \emph{Judgment and
Decision Making}, \emph{6}(6), 552--564.

\leavevmode\hypertarget{ref-Maxwell2018}{}%
Maxwell, S. E., Delaney, H. D., \& Ken, K. (2018). \emph{Designing
experiments and analyzing data: a model comparison perspective} (3rd
edition, pp. 611--612). New York: Routledge.

\leavevmode\hypertarget{ref-Mazar2008}{}%
Mazar, N., Amir, O., \& Ariely, D. (2008). The Dishonesty of Honest
People: A Theory of Self-Concept Maintenance. \emph{Journal of Marketing
Research}, \emph{45}(6), 633--644.
\url{https://doi.org/10.1509/jmkr.45.6.633}

\leavevmode\hypertarget{ref-McElreath2018}{}%
McElreath, R. (2018). \emph{Statistical rethinking: A bayesian course
with examples in R and stan}.
\url{https://doi.org/10.1201/9781315372495}

\leavevmode\hypertarget{ref-Peirce2019}{}%
Peirce, J., Gray, J. R., Simpson, S., MacAskill, M., Höchenberger, R.,
Sogo, H., \ldots{} Lindeløv, J. K. (2019). PsychoPy2: Experiments in
behavior made easy. \emph{Behavior Research Methods}, \emph{51}(1),
195--203. \url{https://doi.org/10.3758/s13428-018-01193-y}

\leavevmode\hypertarget{ref-Peirce2008}{}%
Peirce, J. W. (2008). Generating stimuli for neuroscience using
PsychoPy. \emph{Frontiers in Neuroinformatics}, \emph{2}.
\url{https://doi.org/10.3389/neuro.11.010.2008}

\leavevmode\hypertarget{ref-Rachlin2006}{}%
Rachlin, H., \& Jones, B. (2006). Social Discounting.
\emph{Psychological Science}, \emph{17}(4), 283--286.

\leavevmode\hypertarget{ref-R2019}{}%
R Core Team. (2019). \emph{R: A language and environment for statistical
computing}. Vienna, Austria: R Foundation for Statistical Computing.
Retrieved from \url{https://www.R-project.org/}

\leavevmode\hypertarget{ref-Schad2019}{}%
Schad, D., Betancourt, M., \& Vasishth, S. (2019). Toward a principled
Bayesian workflow in cognitive science. Retrieved from
\url{osf.io/b2vx9}

\leavevmode\hypertarget{ref-Sip2015}{}%
Sip, K. E., Smith, D. V., Porcelli, A. J., Kar, K., \& Delgado, M. R.
(2015). Social closeness and feedback modulate susceptibility to the
framing effect. \emph{Social Neuroscience}, \emph{10}(1), 35--45.
\url{https://doi.org/10.1080/17470919.2014.944316}

\leavevmode\hypertarget{ref-GSS2018}{}%
Smith, T., Davern, M., Freese, J., \& Hout, M. (2018). GSS Data Explorer
\textbar{} NORC at the University of Chicago. Retrieved from
\url{https://gssdataexplorer.norc.org/variables/5067/vshow}

\leavevmode\hypertarget{ref-Rstan2018}{}%
Stan Development Team. (2018). RStan: The R interface to Stan. Retrieved
from \url{http://mc-stan.org/}

\leavevmode\hypertarget{ref-Tough2017}{}%
Tough, H., Siegrist, J., \& Fekete, C. (2017). Social relationships,
mental health and wellbeing in physical disability: A systematic review.
\emph{BMC Public Health}, \emph{17}(1), 1--18.
\url{https://doi.org/10.1186/s12889-017-4308-6}

\leavevmode\hypertarget{ref-Vanberg2008}{}%
Vanberg, C. (2008). Why Do People Keep Their Promises? An Experimental
Test of Two Explanations1. \emph{Econometrica}, \emph{76}(6),
1467--1480. \url{https://doi.org/10.3982/ECTA7673}

\leavevmode\hypertarget{ref-Vuorre2018}{}%
Vuorre, M., \& Bolger, N. (2018). Within-subject mediation analysis for
experimental data in cognitive psychology and neuroscience.
\emph{Behavior Research Methods}, \emph{50}(5), 2125--2143.
\url{https://doi.org/10.3758/s13428-017-0980-9}

\leavevmode\hypertarget{ref-Xu2016}{}%
Xu, S., Pan, Y., Wang, Y., Spaeth, A. M., Qu, Z., \& Rao, H. (2016).
Real and hypothetical monetary rewards modulate risk taking in the
brain. \emph{Scientific Reports}, \emph{6}.
\url{https://doi.org/10.1038/srep29520}


\end{document}
