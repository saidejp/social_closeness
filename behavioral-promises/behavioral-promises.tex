% !TeX program = pdfLaTeX
\documentclass[smallextended]{svjour3}       % onecolumn (second format)
%\documentclass[twocolumn]{svjour3}          % twocolumn
%
\smartqed  % flush right qed marks, e.g. at end of proof
%
\usepackage{amsmath}
\usepackage{graphicx}
\usepackage[utf8]{inputenc}

\usepackage[hyphens]{url} % not crucial - just used below for the URL
\usepackage{hyperref}
\providecommand{\tightlist}{%
  \setlength{\itemsep}{0pt}\setlength{\parskip}{0pt}}

%
% \usepackage{mathptmx}      % use Times fonts if available on your TeX system
%
% insert here the call for the packages your document requires
%\usepackage{latexsym}
% etc.
%
% please place your own definitions here and don't use \def but
% \newcommand{}{}
%
% Insert the name of "your journal" with
% \journalname{myjournal}
%

%% load any required packages here



\usepackage{booktabs}
\usepackage{longtable}
\usepackage{array}
\usepackage{multirow}
\usepackage{wrapfig}
\usepackage{float}
\usepackage{colortbl}
\usepackage{pdflscape}
\usepackage{tabu}
\usepackage{threeparttable}
\usepackage{threeparttablex}
\usepackage[normalem]{ulem}
\usepackage{makecell}
\usepackage{xcolor}

\begin{document}

\title{Nice people or potential cooperators when keeping promises }
 \subtitle{An experimental and Bayesian account for two explanations} 

    \titlerunning{Nice people or potential cooperators}

\author{  Said Jiménez \and  Kamila E. Sip \and  Roberto E. Mercadillo \and  Diego Angeles-Valdez \and  Jairo Muñoz-Delgado \and  Juan J. Sánchez-Sosa \and  Eduardo A. Garza-Villarreal \and  }

    \authorrunning{ S. Jiménez, et al. }

\institute{
        Said Jiménez \at
     Department of Psychology, National University of México (UNAM) \\
     \email{\href{mailto:said.ejp@gmail.com}{\nolinkurl{said.ejp@gmail.com}}}  %  \\
%             \emph{Present address:} of F. Author  %  if needed
    \and
        Eduardo A. Garza-Villarreal \at
     Institute of Neurobiology, UNAM \\
     \email{\href{mailto:egarza@gmail.com}{\nolinkurl{egarza@gmail.com}}}  %  \\
%             \emph{Present address:} of F. Author  %  if needed

\date{Received: date / Accepted: date}
% The correct dates will be entered by the editor


\maketitle

\begin{abstract}
Promises increase cooperation between non-genetically related
individuals, whereas lies and betrayals occur daily even in close social
relationships. Despite this, the effect of social closeness on the
decision to keep or break a promise has not been thoroughly studied. We
conducted an experiment in which subjects could freely decide if they
broke or kept a promise to three partners with different levels of
social closeness: null (computer), low (strange) and high (friend). If
subjects kept their word with the computer, it provides evidence in
favor of intrinsic motivation (comply to do the correct thing), while
extrinsic motivation would be supported (comply to facilitate future
cooperation). Using Hierarchical Bayesian Modeling, we found that as
social closeness increases, the probability of breaking promises
decreases monotonically. The evidence ratio in favor of the hypothesis
stranger \textless{} computer was 7.47, and friend \textless{} stranger
was 8.64. Furthermore, using Bayesian Cumulative Modeling we found that
subjects were very consistent between their promises and their choices.
Although it was more probable that our subjects kept their promises to
any of the three partners, there was a proportion of transgressions to
the computer, whose inference excludes zero with a 95\% posterior
probability. Also, our subjects never broke the promise to their
friends. Our results suggest that subjects were predominantly ``nice
people'' because they kept their promises even to partners without
social closeness. However, if the subjects were not certain that their
partner will also be a ``potential cooperator'', dishonesty emerged in
the form of broken promises.
\\
\keywords{
        Deception \and
        Hierarchical Bayesian Models \and
        Bayesian Cumulative Models \and
        Social closeness \and
        Promises \and
    }


\end{abstract}


\def\spacingset#1{\renewcommand{\baselinestretch}%
{#1}\small\normalsize} \spacingset{1}


\hypertarget{introduction}{%
\section{Introduction}\label{introduction}}

Humans are social animals and we obtain survival advantages from proper
socialization. Historically, belonging to a social group has provided
protection and guaranteed access to resources such as food, and
currently, there is evidence that belonging to a social group has a
positive effect on longevity, physical and mental health (Tough,
Siegrist, and Fekete 2017; Holt-Lunstad 2018). The interaction between
social closeness and cooperation seems to be a key feature to the
formation of large groups of individuals without a genetic relationship,
and such groups form the basis of communities, societies, and nations,
arguably constituting one of the most fundamental conditions for human
survival (Fehr and Fischbacher 2003; Fehr and Schurtenberger 2018). A
recent meta-analysis showed that communication, regardless of the
medium, is the most important element for cooperation between
individuals (Balliet 2010). In parallel, there is evidence that
following communication sessions between completely unknown individuals,
there is an increase in subjective social closeness(Aron et al. 1997).
These findings suggest the importance of the relationship between
communication, cooperation, and social closeness.

One of the forms of communication that have received the most attention
in psychological game theory is ``the promise''. In an interaction
between a sender and a recipient, it is believed that the sender's
promises influence the recipient's beliefs, influencing their ability to
trust and cooperate (Charness and Dufwenberg 2006; Vanberg 2008).
Commonly, however, there are situations where promises are broken, trust
is betrayed and people are deceived. The General Social Survey (GSS) in
2018 showed that approximately 11\% of people surveyed responded that
they have had sexual intercourse with someone different to their partner
while they were married (Smith et al. 2018). In economic and social
terms, the United Nations Organization has stressed that deception
(corruption) has a global cost of at least 5 percent of the annual gross
domestic product (GDP), and it is a generator of conflict and
instability in all nations (UN 2018). Hence, breaking promises may
greatly affect our own social groups and society itself.

Despite the widespread presence of interactions between human belonging
to the same group, it is not known how social closeness between
participants can affect keeping or breaking promises. This is important
because it has been pointed out that lies and betrayals of trust occur
daily even in close relationships such as friends, coworkers, or
classmates, and their consequences can be serious, from loss of the
relationship to work loss and damage to reputation (DePaulo et al.
2004). The two main motivations that have been pointed out in the
literature regarding keeping promises are: instrumental and intrinsic
(Baumgartner et al. 2009). The instrumental suggests that promises are
kept to make future cooperation easier while the intrinsic suggests that
the promises are kept to do what is ``morally right''. In the
laboratory, the betrayals of trust have been explicitly studied in
experiments where subjects have incentives to lie and break promises.
However, the vast majority of studies have been conducted in people
without any social closeness (strangers) (Gneezy 2005; Gneezy,
Rockenbach, and Serra-Garcia 2013; Baumgartner et al. 2009; Baumgartner,
Gianotti, and Knoch 2013; Charness and Dufwenberg 2006; Mazar, Amir, and
Ariely 2008; Fischbacher and Föllmi-Heusi 2013). Some studies have
addressed deception in close interpersonal relationships using
self-report measurements or rather explore the development of deception
detection skills in same-sex friends, with clear limitations due to the
social desirability bias (DePaulo and Kashy 1998; DePaulo et al. 2004;
Anderson, DePaulo, and Ansfield 2002).

In this study, we wanted to understand the effects of social closeness
on keeping or breaking promises and cooperation, by means of an
experimental trust game with three partners with different levels of
social closeness. For this, subjects performed a standard trust game in
pairs with three phases: 1) the trustee makes a promise to pay half of
his earnings regardless of who his/her investor; 2) the investor
receives the promise and decides if he/she invests his/her initial
budget; 3) in case the investor has given the budget, the trustee faces
the decision to pay or not to pay half of his earnings (keep or break
promise). To evaluate the effect of social closeness, our subjects
participate in the role of trustee with three partners with different
levels of social closeness (zero, low and high) in the role of the
investor: a computer, a stranger, and a close friend. Our main
pre-registered hypothesis was:

\begin{enumerate}
\def\labelenumi{\arabic{enumi}.}
\tightlist
\item
  Social closeness will reduce the decision of breaking the promise.
  According to the instrumental motivation, we expect that subjects keep
  their promises with friends for the purpose of facilitating future
  cooperation, which probably can be extended even beyond the trials in
  the experiment. We also expect, although to a lesser extent, that the
  subjects keep promises to strangers, with the purpose of facilitating
  cooperation during the experiment. Finally, we anticipate that the
  participants break the promises to the computer because it is a
  partner without any social closeness and they could not ensure
  cooperation in future trials. It should be noted that if the
  participants keep the promises to the computer, evidence would be
  given in favor of intrinsic motivation.
\end{enumerate}

Our secondary exploratory hypotheses were:

\begin{enumerate}
\def\labelenumi{\arabic{enumi}.}
\setcounter{enumi}{1}
\item
  There will be an effect of social closeness on cooperation, regardless
  of promises. According to the instrumental motivation, the payment
  will be greater for partners with whom the subject anticipates greater
  cooperation in the future (friend \textgreater{} stranger
  \textgreater{} computer).
\item
  We hypothesize that the subjects will cooperate according to their
  degree of commitment expressed through promises.
\end{enumerate}

\hypertarget{methods}{%
\section{Methods}\label{methods}}

\hypertarget{subjects}{%
\subsection{Subjects}\label{subjects}}

We recruited 45 subjects (15 male) from the National Autonomous
University of Mexico, age range from 19 to 33 years, their minimum
educational level were bachelor and right handed (for MRI). Subjects
were asked to come to the study with a ``close'' friend, who fulfilled
the following criteria: 1) he/she was matched by sex, 2) did not have a
family bond and 3) was not a person with a sentimental or sexual
relationship. From this sample (n = 45), 30 subjects (15 men) performed
the task in a magnetic resonance imaging (MRI) scanner, however, their
image data was not analyzed for the present work.

\hypertarget{task}{%
\subsection{Task}\label{task}}

All subjects performed an adaptation of the trust game with promises,
using hypothetical monetary rewards (Baumgartner et al. 2009). The
variation with respect to the original task was that, in our experiment,
three partners with three levels of social closeness were presented in
the role of the investor: computer (no closeness), strange (low
closeness) and friend (high closeness), while the participant acted as
the trustee. The task was programmed in PsychoPy2 version 1.84.2 (Peirce
et al. 2019; Peirce 2008) and consisted of 24 trials between two
players: trustee and investor. The trustee originally had 0 Mexican
pesos and the investor had 2 pesos, the investor was presented with the
opportunity to give his/her money to the trustee or keep it. If the
investor gave his/her money, it was multiplied by 5 pesos, so that the
trustee had 10 pesos. Finally, the trustee had to decide if he/she
wanted to pay half to the investor or keep the 10 pesos. The structure
described was repeated in 24 trials, however, in 4 of the trials the
trustee could send a promise to the investor, the promises were that
he/she would: always, mostly, sometimes, or never pay back. Each promise
was valid for three trials, so that 12 of the trials had the effect of
the promise and the other 12 did not. Each partner performed the role of
investor in 8 of 24 total trials, however, both the promises and social
closeness conditions were presented to the trustee in pseudorandom
order.

Since our main interest was the trustee's behavior and to avoid
unbalanced responses from investors, the investor was always the
computer although the trustee was shown to have 3 types of investors,
and the investors' decisions were programmed \emph{a priori} to randomly
give their amount in 6 trials and withhold the amount in 2 trials. The
covert story for all our subjects was that they would be playing in
real-time with their friend (whom they brought to the study), the
stranger (was told it would be another same-sex unknown person) and the
computer. After the study ended, the participants were debriefed about
our deception. A diagram of the the experimental task is shown in Figure
1.

\begin{figure}

{\centering \includegraphics[width=0.8\linewidth]{/Users/saidjimenez/Documents/R/github_Said/social_closeness/Manuscript/figures/task} 

}

\caption{Trust game with promises, each box from left to right represents a screen that was shown to the subjects sequentially. Section A1 corresponds to an example of trials without promises and section A2 to trials with promises.}\label{fig:figA}
\end{figure}

Promise phase in part A1 indicated that in the following three trials
they could decide without the effect of the promise. In A2, the subject
was asked to decide between 1) always, 2) mostly, 3) sometimes or 4)
never pay back. This pattern was repeated. In the
anticipation/assignation phase, subjects were told who their partner was
for that trial (computer, stranger or friend) and was given the message
that their partner was making their decision. Later in the decision
phase, the subject was informed if his partner invested his \$2 or not,
he was reminded of his promise level (if they were trails with promises)
and, in case the partner had invested, he was asked to decide whether to
pay back or not. Finally, payments for each trial were shown in the
feedback phase and the sequence was repeated until the task finished.

\hypertarget{procedure}{%
\subsection{Procedure}\label{procedure}}

Subjects came to the lab with a same-sex friend considered close by him
or her. It was emphasized that they must not have a romantic or family
relationship with their friends, to try to exclude the effect on
cooperation due to a consanguineous relationship or sexual attraction.
Also, to ensure that the subjects and their companions had a similar
degree of social closeness to each other, both responded the ``Inclusion
of the Other in Self'' IOS scale (Aron et al. 1997) without observing
their partner responses. The scale consists of seven pairs of circles
that vary in the degree of overlap between them, the respondent must
select the pair of circles that best represents the subjective closeness
to his partner. Subjects were trained in the main task by two
researchers, the game was first explained verbally, and then the
computer interface was presented on a portable computer. They were told
that they would make their decisions with four buttons on the computer
keyboard, which corresponded to the levels of promises and with only two
of these to make their decision to pay back or not. Friends heard the
explanation, they were told that they would make their decisions in a
separate room on another computer even though they were not. Subjects
performed between 3 and 6 practice tests together with the researcher
who exemplified the course of the game when the investor gave his
budget. When the subjects were ready, we proceeded to take them without
their companions to another room (or MRI scanner it that was the case)
where they would perform the task.The subjects began the task believing
that they would really play with the three partners of different social
closeness. In the other room, we debriefed the companion about the
deception. When the subject finished the task, they were debriefed.
Although no participant exercised their right, both subjects and their
friends were told that they could withdraw their participation and
informed consent if they considered disagreeing with any of the
manipulations made by the researchers.

\hypertarget{bayesian-modeling}{%
\subsection{Bayesian Modeling}\label{bayesian-modeling}}

All models presented below were programmed in R via the \textbf{brms}
package (Bürkner 2017; R Core Team 2019), which performs the inference
using sampling by Markov chain Monte Carlo through \textbf{Stan} (Stan
Development Team 2018). For each model, the posteriors distributions of
all parameters were approximated with four chains of 2000 iterations
each, the first 1000 iterations of each chain were discarded (burning
period), for a total of 4000 post burning samples. Models convergence
was evaluated through visual inspection of the chains and calculation of
the \(\hat{R}\) statistic, which for all parameters was 1, that can be
interpreted as convergence. Vaguely informative prior distributions were
used for the parameters of interest in the models, which allows the data
to dominate the inference, also assumes as little as possible regarding
the nature of the phenomenon, which could be adequate for the current
state of evidence in the problem we are studying (McElreath 2018).

\hypertarget{results}{%
\section{Results}\label{results}}

\hypertarget{descriptives}{%
\subsection{Descriptives}\label{descriptives}}

We analyzed 810 decisions about paying back or not (18 per subject), as
well as 180 promises (4 per subject). 69\% of the decisions, were pay
back, while the proportion of the promises selected were never = 5.4\%,
sometimes = 16.2\%, mostly = 43.8\% and always = 40.2\%. Figure 2 shows
the payment proportion depending on the levels of promises and partners.
Breaking promises, meaning the subject had promised always and
subsequently decides not to pay back, occurred only in 12.4\% of trials
(19 times they did not pay back of 153 when promised always), of which,
8.50\% (13/153) were trials with the computer, 3.92\% (6/153) with the
stranger and 0\% (0/153) with the friend. It is essential to note that
breaking the promise was not the most frequent decision, when they
promised always and had the computer as a partner, they decided to pay
back in 91.5\% of trials.

\begin{figure}

{\centering \includegraphics[width=0.8\linewidth]{behavioral-promises_files/figure-latex/fig2-1} 

}

\caption{Pay back proportion by partners and promises levels, the x-axis represents all decisions made to each partner, the different shades of blue shows the proportions in which the subjects paid back or not. The y-axis presents how these proportions vary according to the promises made.}\label{fig:fig2}
\end{figure}

\hypertarget{social-closeness-between-partners}{%
\subsubsection{Social closeness between
partners}\label{social-closeness-between-partners}}

To assess social closeness, 30 (15 men) of our subjects and their
friends responded to the IOS scale (Aron et al. 1997; Fareri and Delgado
2014; Sip et al. 2015), which consists of seven pairs of circles that
vary in the degree of overlap and represent the social closeness that an
individual perceives with respect to the other. A Hierarchical and
Cumulative Bayesian Model (See section: Commitment expressed in the
promises), supports the hypothesis that there are no differences in
subjective social closeness between subjects and their friends, with a
posterior evidence ratio of 2.74 in its favor.

\hypertarget{hierarchical-bayesian-modeling}{%
\subsection{Hierarchical Bayesian
Modeling}\label{hierarchical-bayesian-modeling}}

\hypertarget{effect-of-social-closeness-on-breaking-promises}{%
\subsubsection{Effect of social closeness on breaking
promises}\label{effect-of-social-closeness-on-breaking-promises}}

To evaluate the hypothesis that social closeness would reduce the
behavior of breaking the promise, we filtered all trials in which
subjects promised that they would always pay back and then decided not
to pay. Subsequently, Bayesian inference was used to assess the effect
of partners on the decision to break the promise at the individual and
population level. For this purpose, a Hierarchical Bayesian Model was
carried out that assumes that the uncertainty in partners effect on the
decision to break the promise varies depending on each individual,
however, it also assumes that these variations belong to common
population distributions (Gelman and Hill 2006).

\[
\begin{aligned}
&y_i \sim \mathrm{Bernoulli}(\theta_i) \\
&\mathrm{logit(\theta_i)} = \mathbf{X}\beta  +  \mathbf{Z}u
\end{aligned}
\]

In the previous model, the decision to break the promise \(y_i\) comes
from a Bernoulli distribution with probability \(\theta_i\), the goal of
the Hierarchical Model is to predict each decision through the linear
combination of the effects of each partner, transformed by its inverse
link function \(\mathrm{logit}\) (Bürkner 2017). In this model,
\(\beta\) and \(u\) are coefficients at the population level and
individual level respectively, while \(\mathbf{X}\), \(\mathbf{Z}\) are
their corresponding design matrices. In this case, the population
coefficients correspond to the presence of partners with no social
closeness \(\beta^{computer}\), low \(\beta^{stranger}\) and high
\(\beta^{friend}\).

Figure 3 shows the posterior probability of breaking the promise
depending on the partner, circles correspond to the medians of the
distributions of the posterior estimates effects, the thick bar, and the
thin bar correspond to the interval of 50\% and 95\% posterior
probability, respectively. It is clear that the probability of breaking
the promise decreases as social closeness increases, however, we
calculate the reasons for evidence for the following hypotheses:

\begin{itemize}
\item
  There is a greater probability to break the promise to the computer
  than the stranger.
\item
  There is more probability to break the promise to the stranger
  compared to a friend.
\end{itemize}

The evidence ratio, which is the ratio between the posterior probability
of the mentioned hypotheses and their corresponding alternative
hypotheses was 7.47 and 8.64 respectively in favor of the previous
hypotheses. As seen in Figure 5 in the estimate for
\(\beta^{computer}\), no social closeness, we have a 95\% posterior
probability that the parameter for promise-breaking is between 4-38\%,
although it is a small proportion, it clearly excludes the probability
that promises are not broken.

\begin{figure}

{\centering \includegraphics[width=0.8\linewidth]{behavioral-promises_files/figure-latex/fig3-1} 

}

\caption{Inference regarding the probability of breaking the promise depending on social closeness. The x-axis shows the posterior probability of breaking the promise, and the y-axis indicates the three partners. The dotted line shows the probability that breaking the promise would be zero.}\label{fig:fig3}
\end{figure}

\hypertarget{effects-of-promises-and-social-closeness-on-cooperation}{%
\subsubsection{Effects of promises and social closeness on
cooperation}\label{effects-of-promises-and-social-closeness-on-cooperation}}

To evaluate the effect of the experimental conditions on the decision to
pay back, we used all trials were partners invest. Again, a Hierarchical
Bayesian Model was fit, which assumes that the promises and partners
have an effect that varies for each individual, however, it also assumes
that these variations belong to common population distributions. The
model estimates that the probability \(\theta_i\) of the decision to pay
back is based on the effect of the presence of promises
\(\beta^{promise}\), as well as partners with low \(\beta^{stranger}\)
and high \(\beta^{friend}\) social closeness.

Figure 4 shows the posterior distributions of the \(\beta\) coefficients
in the logit scale, the center line represents the median of the
distribution and the shaded area corresponds to the interval of 50\%
posterior probability. As shown, more than 95\% of the posterior density
of the population coefficients is greater than 0, which shows strong
evidence of the effect of experimental conditions on the decision to pay
back, although the presence of promises and the stranger clearly
increase the probability of paying, the presence of the friend is the
condition that has the greatest effect on this behavior. Posterior
distributions clearly show that the probability of cooperation increases
as social closeness does.

\begin{figure}

{\centering \includegraphics[width=0.8\linewidth]{behavioral-promises_files/figure-latex/fig4-1} 

}

\caption{Posterior estimates in logit scale. We show posteriors distributions of the hierarchical Bayesian model coefficients for the effect of experimental conditions on the decision to cooperate. The x-axis shows the magnitude of the logit scale coefficients, and the y-axis shows the experimental conditions associated with that effect.}\label{fig:fig4}
\end{figure}

Figure 5 shows the posterior predictive distribution in contrast to the
proportion of decisions to each partner during the promises phase, and
each panel corresponds to one of the first twelve subjects. The
responses on the dotted line would indicate that the subject paid back
random to that partner. Posterior predictive distribution simulates
observations of the model and compares them with the actual data, it
helps us to identify if the model is sufficiently close to the process
that generated the data (Schad, Betancourt, and Vasishth 2019; Lee and
Wagenmakers 2014). It can be seen that there is a great correspondence
between the subject's responses and predictions of the model. Even in
cases where the model is ``wrong'' (for example, subject 1), the
observed response is in the range of a predicted standard deviation,
which gives credibility to estimates.

\begin{figure}

{\centering \includegraphics[width=1\linewidth,height=0.42\textheight]{behavioral-promises_files/figure-latex/fig5-1} 

}

\caption{Posterior predictive over pay back rate. Correspondence between payment proportions and model predictions depending on the game partner for the first twelve subjects in the sample. The x-axis pointed the three playmates when the subjects had made a promise, while the y-axis indicates the proportion of decisions in which subjects payback.}\label{fig:fig5}
\end{figure}

\hypertarget{effects-of-social-closeness-on-cooperation-varying-by-promises}{%
\subsubsection{Effects of social closeness on cooperation varying by
promises}\label{effects-of-social-closeness-on-cooperation-varying-by-promises}}

In this model, it is assumed that social closeness has an effect that
varies for each level of promise, which implies that there are levels of
promise that are more sensitive than others to the effect of the
partner's on the decision to pay back. Again, a Hierarchical Bayesian
Model was made to estimate the effect of partners at the population
level and the variations depending on the promise levels. Table 1
summarizes the subsequent distributions of the model coefficients in the
logit scale, including point estimates, standard errors and Bayesian
Credibility Intervals of 95\%. An estimate similar to the previous
models can be observed, with strong evidence of the effect of social
closeness on the decision to pay back. Although the credibility range
for the friend's effect includes 0, the evidence ratio that the effect
is greater than zero is 15.81 with 94\% of posterior probability.

\begin{figure}

{\centering \includegraphics[width=0.8\linewidth]{behavioral-promises_files/figure-latex/fig6-1} 

}

\caption{Posterior predictive over pay back rate, social closeness varying effects by promise levels. Correspondence between payment proportions and model predictions depending on the game partner at the group level. The x-axis pointed the three playmates when the subjects had made a promise, while the y-axis indicates the proportion of decisions in which subjects payback.}\label{fig:fig6}
\end{figure}

On the other hand, Figure 6 shows the posterior predictive distribution,
compared to the payment rates of all individuals to the different
partners and their variation by the level of promise. With the exception
of the promise never, a monotonic positive effect of partners is
observed at all levels of promise. However, we also observe how the
effect of social closeness varies depending on the strength of the
promise, mainly for the decisions towards the computer.

\begin{table}[t]

\caption{\label{tab:tabla}Posterior coefficients estimates}
\begin{tabular}{lrrrr}
\toprule
\multicolumn{1}{c}{ } & \multicolumn{2}{c}{ } & \multicolumn{2}{c}{95 \% CI} \\
\cmidrule(l{3pt}r{3pt}){4-5}
Term & Estimate & Est.Error & Lower & Upper\\
\midrule
Intercept & -0.153 & 0.536 & -1.005 & 0.617\\
Stranger & 0.829 & 0.410 & 0.128 & 1.470\\
Friend & 2.084 & 1.248 & -0.151 & 3.975\\
\bottomrule
\end{tabular}
\end{table}

\hypertarget{cumulative-bayesian-modeling}{%
\subsection{Cumulative Bayesian
Modeling}\label{cumulative-bayesian-modeling}}

\hypertarget{commitment-expressed-in-promises}{%
\subsubsection{Commitment expressed in
promises}\label{commitment-expressed-in-promises}}

In the original study of promises, authors chose to divide their sample
into two according to the hierarchical clustering technique with Ward's
method (Baumgartner et al. 2009). In this way they obtained two sets of
participants that were different in their payment rates, despite the
fact that they both made very high promises. For this reason, the
authors named the group that paid little as ``dishonest'' and the group
that paid a lot as ``honest''. In a similar exercise, in the present
study we performed the hierarchical grouping technique to obtain a
solution for two groups and we found two similar sets in n that we call
the group ``Low'' and ``High'' (Low = 25, High = 20), with quite
different payment ratios (Low = 58\%, High = 83\%), and a 95\%
confidence interval of 20\% to 32\% in the difference favor of the High
group.

Although it seems a similar result to that reported in that paper
(Baumgartner et al. 2009), we explore the pattern of promises of both
groups to determine if it was possible to classify our subjects as
honest and dishonest. If we hypothesize that the commitment to pay would
be reflected in the level of promises selected, a group of dishonest
people could generate in their partners the belief that they will pay by
choosing a high level of promises (mostly or always) and, later,
betraying that trust when deciding not to pay back. In order to explore
whether the groups represent populations that do not differ in the level
of commitment expressed in the promises, we use a Cumulative Bayesian
Model, which assumes that the levels of promises are an observed ordinal
variable \(Y\) that originates from the categorization of a continuous
latent variable \(\tilde{Y}\), for this case, the expressed commitment
to pay back (Bürkner and Vuorre 2019). The degree to which the subjects
of the High group differ from the Low group, in normal standard
deviations (\(z\)-values), on the latent scale of \(\tilde{Y}\), has a
point estimate of 0.95, which implies that the High group has 0.95
z-values greater commitment to pay back than the Low group. The 95\%
Bayesian Credibility Interval indicates that the High group is between
0.40 to 1.51 \(z\)-values of difference from Low. So we can conclude
with at least a 95\% probability that people belonging to the High group
expressed in their promises a greater commitment to pay back than the
subjects of the Low group. If we look at Figure 7, the probability of
choosing the different levels of promises varies depending on the group,
the one who had the highest percentage of decisions to pay back is more
likely to choose always (High group), while the group that had the
lowest percentage of decisions to pay back are more likely to choose
mostly (Low group). According to our data, we could not justify the
classification of our groups according to their honesty or dishonesty,
at least not with the technique used in Baumgartner et al. (2009). Since
people were quite consistent with keeping what they promised to pay.

\begin{figure}

{\centering \includegraphics[width=0.8\linewidth]{behavioral-promises_files/figure-latex/fig7-1} 

}

\caption{Promise choices by group. We show posterior probability estimates regarding the level of promise depending on the group. The Low group (which paid 58/100 of the trials) is more likely to promise mostly, while the High group (which paid 83/100 of the trials) is more likely to promise always.}\label{fig:fig7}
\end{figure}

\hypertarget{discussion}{%
\section{Discussion}\label{discussion}}

The goal of this study was to evaluate the effect of social closeness on
keeping and breaking promises, as well as on cooperation in a trust
game. Our results give evidence that zero social closeness increases the
probability of breaking the promise and it decreases monotonically as
social closeness increases. Likewise, as social closeness increases, so
does cooperation; high social closeness has an effect on decisions that
surpasses even those that all other experimental conditions. The most
interesting finding was that when subjects expressed high commitment to
cooperate (through choosing that they would always pay back), the
probability that they comply is also quite high. However, there is a
small but consistent proportion of transgressions according to our
inferences.

To our knowledge, this is the first experimental study that incorporates
social closeness as a predictor of the decision to break promises with
socially relevant partners. In previous studies, the participants remain
anonymous during the course of the tasks (Baumgartner et al. 2009),
their measurements to evaluate transgressions are self-reported (DePaulo
and Kashy 1998; DePaulo et al. 2004), they do not directly quantify the
breaking of promises (Vanberg 2008; Charness and Dufwenberg 2006), or as
we will discuss later, they use heterogeneous measures of social
closeness (Glaeser et al. 2000). Our results suggest that the mere fact
of considering that subjects play with a human diminishes the
probability of breaking the promise. It is more likely to break the
promise to the computer than to the stranger, even though that partner
was not known. Also, zero social closeness is important to include
because it allows us to explore the intrinsic motivation to keep the
promises. If humans keep their word because it is morally correct, we
would anticipate that the probability of breaking a promise was very low
in the three partners and, particularly, the inference regarding the
probability of breaking the promise to the computer would include zero.
However, we found that, although low, the probability was greater than
zero, which provides evidence that seems to contradict, at least in some
degree, the hypothesis of intrinsic motivation. This means that we
cannot exclude the possibility that the subjects keep promises mainly
due to moral motivation because, even in the case of the computer, the
subjects kept a large proportion of promises. We can, however, exclude
that this is their \emph{only} motivation.

High social closeness allows us to explore the instrumental motivation
to keep promises (Baumgartner et al. 2009). In our study, subjects were
more likely to break the promise to the stranger than to the friend. If
humans keep their word to facilitate cooperation in the future, we would
anticipate that there would be less of a chance to break the promise
with a partner of high social closeness (with whom they are very likely
to cooperate, even after the experiment) and, in fact, the promise to
friends was not broken on any occasion. The findings concerning
motivations could be explained by the social norm called ``conditional
cooperation'', which indicates that the belief that other people
cooperate at high levels also induces high levels of cooperation (Fehr
and Schurtenberger 2018). Thus, the uncertainty regarding the computer's
decisions could explain the probability of breaking the promise.
Similarly, the information that the subjects have regarding their
friend's behavior, even before the experiment, could explain the high
levels of cooperation and keeping of promises towards the friend.

Cooperation has been studied with diverse experimental manipulations. In
one study, the contribution in a ``public goods game'' was evaluated
during several trials in a group made up of the same individuals
(partners), and compared to another group of new different subjects for
each trial (strangers) (Croson 1996). In the condition of partners, the
cooperation was greater because it was a stable group compared to the
group of strangers. The mentioned study does not give details regarding
the recruitment of participants, thus we could assume that even in the
condition of partners these were individuals who did not consider
themselves socially close. Another study showed an increase in
cooperation in a trust game when individuals were socially close
(Glaeser et al. 2000). In this study, subjects knew each other and the
researchers carefully measured several variables regarding their social
connection such as the number of friends in common and years of
relationship. They found greater cooperation depending on the degree of
social connection. However, the study did not control for social
closeness as some individuals who arrived together were allowed to
perform the task together whereas others were paired differently. This
limitation allowed social closeness between partners to be
heterogeneous, allowed for romantic relationships and assumed that two
individuals who arrived together to class were considered close to each
other. To try to homogenize social closeness between our subjects and
their partners, they performed the task with a same-sex friend
considered to be close, emphasizing their companion could not be a
romantic or sexual partner. Bayesian analysis of the IOS scale showed
evidence in favor that the degree of social closeness between 30 of our
45 subjects and their friends was the same. Although results of the two
mentioned studies also found that social closeness enhance cooperation
(Croson 1996; Glaeser et al. 2000), it would be valuable to homogenize
both the performed tasks and the statistical procedure used to establish
similarity in effects magnitudes.

Our methodological contribution to the field is the use of Bayesian
inference tools: Hierarchical Models and Cumulative Models. The
Hierarchical Models, allowed us to model the effect of the experimental
conditions on the individual response. These models assume that lower
level observations (e.g.~decisions of each individual) are nested in
higher level units (e.g.~individual subjects). Within-subjects designs
have traditionally been analyzed with repeated measures AN(C)OVA,
however, Hierarchical Models grant the advantages of naturally dealing
with unbalanced data, include categorical and/or numerical predictors,
explicitly incorporate individual variability, among others (Vuorre and
Bolger 2018; Gelman and Hill 2006; McElreath 2018; Bürkner 2017). In our
case, the Hierarchical Model allowed us to capture how the experimental
conditions affected the decisions of each subject and the differences
between them. For example, there were obviously motivated instrumental
subjects such as number 8 or 12 who are very sensitive to the identity
of their investors and modify their cooperation under social closeness.
At the same time, there were notably intrinsic subjects such as 9 and 10
who cooperate all the time regardless of who their partners are. The
Hierarchical Model naturally includes this information for the
estimation of population effects, which, if not considered, would lead
to inaccurate inferences (Bürkner 2017; Gelman and Hill 2006; McElreath
2018). The Cumulative Model allowed us to capture the strength of the
commitment expressed through promises. A very frequent problem that has
been pointed out recently is that analyzing ordinal data with methods
that assume that observations are metric can lead to serious inference
errors (Liddell and Kruschke 2018). The Cumulative Models assumes that
the observed responses come from the categorization of a continuous
latent variable (Bürkner and Vuorre 2019). In our study, we could
identify that groups with different payment proportions, which in other
studies have been called ``dishonest'' and ``honest'' (Baumgartner et
al. 2009; Baumgartner, Gianotti, and Knoch 2013), differ in the
commitment they express through their promises. Considering that the
tools to perform the Cumulative Models are relatively recent, in future
studies, it would be valuable to also use the new alternatives for
inference. Although the Hierarchical and Cumulative models are not tools
exclusively for Bayesian inference, their application from this approach
represents several advantages compared to frequentist statistics.
Classic problems such as multiple comparisons, the decision of when to
stop collecting subjects (stopping rule), or use of planned comparisons
versus post hoc, are not factors that affect the Bayesian approach
(Dienes 2011). In Bayesian inference, hypothesis testing makes formal
use of probability to express the plausibility of theories, and in our
case, we were able to obtain evidence ratios regarding the extent of our
data support hypotheses, regardless if these were proposed \emph{a
priori} or \emph{post hoc}.

\hypertarget{limitations}{%
\subsection{Limitations}\label{limitations}}

A limitation of our study was the use of hypothetical rewards compared
to real rewards. A relatively recent study on decision-making reported
that there is more loss aversion when subjects have real monetary
rewards compared to hypothetical in a risk task (Xu et al. 2016).
However, other works do not report differences between the use of both
types of rewards in self-control, temporal and social discount tasks
(Locey, Jones, and Rachlin 2011; Johnson and Bickel 2006; Rachlin and
Jones 2006). Likewise, it can be argued that our results have
theoretical congruence (Fehr and Schurtenberger 2018) and are in the
same direction as other works that use real money (Croson 1996; Glaeser
et al. 2000). Thus, there are not many reasons to expect that other
types of rewards would modify our results. Another possible limitation
could be the use of multiple trials with each partner. Although the
decision to use repeated measures during our design serves the purpose
of reducing contaminant sources and increasing internal validity
(Maxwell, Delaney, and Ken 2018), the studies could benefit from only
including one trial for each partner to have more clarity regarding the
difference in the intrinsic and instrumental motivations. Play a
one-shot with each partner avoids the possibility that multiple trials
could generate the belief that the computer can also vary its behavior
according to the decisions of the trustee.

\hypertarget{conclusions}{%
\section{Conclusions}\label{conclusions}}

Subjects in our study seemed to keep their promises by a combination of
\emph{intrinsic} and \emph{instrumental} motivations. They can be
regarded as predominantly ``nice people'' because they kept their word
even with investors with no social closeness (computer). However, their
trustworthiness was far from perfect, since there was a small proportion
of betrayals committed towards the computer. If subjects could not be
sure their investor would be a ``potential cooperator'', dishonesty may
have emerged in the form of broken promises. Framing our findings in
terms of the theoretical predictions of the social norm ``conditional
cooperation'', social closeness decreases the probability of breaking
the promise and increases cooperation, and suggests the correspondence
between commitment expressed through promises and subsequent behavior.

\hypertarget{bibliography}{%
\section*{Bibliography}\label{bibliography}}
\addcontentsline{toc}{section}{Bibliography}

\hypertarget{refs}{}
\leavevmode\hypertarget{ref-Anderson2002}{}%
Anderson, D. Eric, Bella M. DePaulo, and Matthew E. Ansfield. 2002.
``The development of deception detection skill: A longitudinal study of
same-sex friends.'' \emph{Personality and Social Psychology Bulletin} 28
(4): 536--45. \url{https://doi.org/10.1177/0146167202287010}.

\leavevmode\hypertarget{ref-Aron1997}{}%
Aron, Arthur, Edward Melinat, Elaine N. Aron, Robert Darrin Vallone, and
Renee J. Bator. 1997. ``The experimental generation of interpersonal
closeness: A procedure and some preliminary findings.''
\emph{Personality and Social Psychology Bulletin} 23 (4): 363--77.
\url{https://doi.org/10.1177/0146167297234003}.

\leavevmode\hypertarget{ref-Balliet2010}{}%
Balliet, Daniel. 2010. ``Communication and cooperation in social
dilemmas: A meta-analytic review.'' \emph{Journal of Conflict
Resolution} 54 (1): 39--57.
\url{https://doi.org/10.1177/0022002709352443}.

\leavevmode\hypertarget{ref-Baumgartner2009}{}%
Baumgartner, Thomas, Urs Fischbacher, Anja Feierabend, Kai Lutz, and
Ernst Fehr. 2009. ``The Neural Circuitry of a Broken Promise.''
\emph{Neuron} 64 (5). Elsevier Ltd: 756--70.
\url{https://doi.org/10.1016/j.neuron.2009.11.017}.

\leavevmode\hypertarget{ref-Baumgartner2013}{}%
Baumgartner, Thomas, Lorena R R Gianotti, and Daria Knoch. 2013. ``Who
is honest and why: Baseline activation in anterior insula predicts
inter-individual differences in deceptive behavior.'' \emph{Biological
Psychology} 94 (1). Elsevier B.V.: 192--97.
\url{https://doi.org/10.1016/j.biopsycho.2013.05.018}.

\leavevmode\hypertarget{ref-Burkner2017}{}%
Bürkner, Paul-Christian. 2017. ``brms : An R Package for Bayesian
Multilevel Models Using Stan.'' \emph{Journal of Statistical Software}
80 (1). \url{https://doi.org/10.18637/jss.v080.i01}.

\leavevmode\hypertarget{ref-Burkner2019}{}%
Bürkner, Paul-Christian, and Matti Vuorre. 2019. ``Ordinal Regression
Models in Psychology: A Tutorial.'' \emph{Advances in Methods and
Practices in Psychological Science} 2 (1): 77--101.
\url{https://doi.org/10.1177/2515245918823199}.

\leavevmode\hypertarget{ref-Charness2006}{}%
Charness, Gary, and Martin Dufwenberg. 2006. ``Promises and
partnership.'' \emph{Econometrica} 74 (6): 1579--1601.
\url{https://doi.org/10.1111/j.1468-0262.2006.00719.x}.

\leavevmode\hypertarget{ref-Croson1996}{}%
Croson, Rachel T.A. 1996. ``Partners and strangers revisited.''
\emph{Economics Letters} 53 (1): 25--32.
\url{https://doi.org/10.1016/S0165-1765(97)82136-2}.

\leavevmode\hypertarget{ref-DePaulo2004}{}%
DePaulo, Bella M, Matthew E Ansfield, Susan E Kirkendol, and Joseph M
Boden. 2004. ``Serious Lies.'' \emph{Basic \& Applied Social Psychology}
26 (2/3): 147--67. \url{https://doi.org/10.1207/s15324834basp2602\&3_4}.

\leavevmode\hypertarget{ref-DePaulo1998}{}%
DePaulo, Bella M., and Deborah A. Kashy. 1998. ``Everyday Lies in Close
and Casual Relationships.'' \emph{Journal of Personality and Social
Psychology} 74 (1): 63--79.
\url{https://doi.org/10.1037/0022-3514.74.1.63}.

\leavevmode\hypertarget{ref-Dienes2011}{}%
Dienes, Zoltan. 2011. ``Bayesian versus orthodox statistics: Which side
are you on?'' \emph{Perspectives on Psychological Science} 6 (3):
274--90. \url{https://doi.org/10.1177/1745691611406920}.

\leavevmode\hypertarget{ref-Fareri2014}{}%
Fareri, Dominic S., and Mauricio R. Delgado. 2014. ``Differential reward
responses during competition against in- and out-of-network others.''
\emph{Social Cognitive and Affective Neuroscience} 9 (4): 412--20.
\url{https://doi.org/10.1093/scan/nst006}.

\leavevmode\hypertarget{ref-Fehr2003}{}%
Fehr, Ernst, and Urs Fischbacher. 2003. ``The nature of human
altruism.'' \emph{Nature} 425 (6960). Nature Publishing Group: 785--91.
\url{https://doi.org/10.1038/nature02043}.

\leavevmode\hypertarget{ref-Fehr2018}{}%
Fehr, Ernst, and Ivo Schurtenberger. 2018. ``Normative foundations of
human cooperation review-article.'' \emph{Nature Human Behaviour} 2 (7).
Springer US: 458--68. \url{https://doi.org/10.1038/s41562-018-0385-5}.

\leavevmode\hypertarget{ref-Fischbacher2013a}{}%
Fischbacher, Urs, and Franziska Föllmi-Heusi. 2013. ``Lies in
disguise-an experimental study on cheating.'' \emph{Journal of the
European Economic Association} 11 (3): 525--47.
\url{https://doi.org/10.1111/jeea.12014}.

\leavevmode\hypertarget{ref-Gelman2006}{}%
Gelman, Andrew, and Jennifer Hill. 2006. \emph{Data analysis using
regression and multilevel/hierarchical models.}
\url{https://doi.org/10.2277/0521867061}.

\leavevmode\hypertarget{ref-Glaeser2000}{}%
Glaeser, Edward, David Laibson, Jose Sceinkman, and Christine Soutter.
2000. ``Measuring Trust.'' \emph{The Quartely Journal of Economics} 115
(3): 811--46.

\leavevmode\hypertarget{ref-Gneezy2005}{}%
Gneezy, Uri. 2005. ``Deception: The role of consequences.''
\emph{American Economic Review} 95 (1): 384--94.
\url{https://doi.org/10.1257/0002828053828662}.

\leavevmode\hypertarget{ref-Gneezy2013}{}%
Gneezy, Uri, Bettina Rockenbach, and Marta Serra-Garcia. 2013.
``Measuring lying aversion.'' \emph{Journal of Economic Behavior and
Organization} 93. Elsevier B.V.: 293--300.
\url{https://doi.org/10.1016/j.jebo.2013.03.025}.

\leavevmode\hypertarget{ref-Holt-Lunstad2018}{}%
Holt-Lunstad, Julianne. 2018. ``Why Social Relationships are Important
for Physical Health: A Systems Approach to Understanding and Modifying
Risk and Protection.'' \emph{Ssrn}, no. October 2017: 1--22.
\url{https://doi.org/10.1146/annurev-psych-122216-011902}.

\leavevmode\hypertarget{ref-Johnson2006}{}%
Johnson, Matthew W, and Warren K Bickel. 2006. ``Within-subject
comparison of real and hypothetical money rewards in delay
discounting.'' \emph{Journal of the Experimental Analysis of Behavior}
77 (2): 129--46. \url{https://doi.org/10.1901/jeab.2002.77-129}.

\leavevmode\hypertarget{ref-Lee2014}{}%
Lee, Michael D., and Eric-Jan Wagenmakers. 2014. \emph{Bayesian
Cognitive Modeling}. \url{https://doi.org/10.1017/cbo9781139087759}.

\leavevmode\hypertarget{ref-Liddell2018}{}%
Liddell, Torrin M., and John K. Kruschke. 2018. ``Analyzing ordinal data
with metric models: What could possibly go wrong?'' \emph{Journal of
Experimental Social Psychology} 79 (November 2017). Elsevier: 328--48.
\url{https://doi.org/10.1016/j.jesp.2018.08.009}.

\leavevmode\hypertarget{ref-Locey2011}{}%
Locey, Matthew L, Bryan A Jones, and Howard Rachlin. 2011. ``Real and
hypothetical rewards in self-control and social discounting.''
\emph{Judgment and Decision Making} 6 (6): 552--64.

\leavevmode\hypertarget{ref-Maxwell2018}{}%
Maxwell, Scott E., Harold D. Delaney, and Kelley Ken. 2018.
\emph{Designing experiments and analyzing data: a model comparison
perspective}. 3rd edition. New York: Routledge.

\leavevmode\hypertarget{ref-Mazar2008}{}%
Mazar, Nina, On Amir, and Dan Ariely. 2008. ``The Dishonesty of Honest
People: A Theory of Self-Concept Maintenance.'' \emph{Journal of
Marketing Research} 45 (6): 633--44.
\url{https://doi.org/10.1509/jmkr.45.6.633}.

\leavevmode\hypertarget{ref-McElreath2018}{}%
McElreath, Richard. 2018. \emph{Statistical rethinking: A bayesian
course with examples in R and stan}.
\url{https://doi.org/10.1201/9781315372495}.

\leavevmode\hypertarget{ref-Peirce2019}{}%
Peirce, Jonathan, Jeremy R. Gray, Sol Simpson, Michael MacAskill,
Richard Höchenberger, Hiroyuki Sogo, Erik Kastman, and Jonas Kristoffer
Lindeløv. 2019. ``PsychoPy2: Experiments in behavior made easy.''
\emph{Behavior Research Methods} 51 (1): 195--203.
\url{https://doi.org/10.3758/s13428-018-01193-y}.

\leavevmode\hypertarget{ref-Peirce2008}{}%
Peirce, Jonathan W. 2008. ``Generating stimuli for neuroscience using
PsychoPy.'' \emph{Frontiers in Neuroinformatics} 2.
\url{https://doi.org/10.3389/neuro.11.010.2008}.

\leavevmode\hypertarget{ref-Rachlin2006}{}%
Rachlin, Howard, and Bryan Jones. 2006. ``Social Discounting.''
\emph{Psychological Science} 17 (4): 283--86.

\leavevmode\hypertarget{ref-R2019}{}%
R Core Team. 2019. \emph{R: A Language and Environment for Statistical
Computing}. Vienna, Austria: R Foundation for Statistical Computing.
\url{https://www.R-project.org/}.

\leavevmode\hypertarget{ref-Schad2019}{}%
Schad, Daniel, Michael Betancourt, and Shravan Vasishth. 2019. ``Toward
a principled Bayesian workflow in cognitive science.''
\url{osf.io/b2vx9}.

\leavevmode\hypertarget{ref-Sip2015}{}%
Sip, Kamila E., David V. Smith, Anthony J. Porcelli, Kohitij Kar, and
Mauricio R. Delgado. 2015. ``Social closeness and feedback modulate
susceptibility to the framing effect.'' \emph{Social Neuroscience} 10
(1): 35--45. \url{https://doi.org/10.1080/17470919.2014.944316}.

\leavevmode\hypertarget{ref-GSS2018}{}%
Smith, Tom, Michael Davern, Jeremy Freese, and Michael Hout. 2018. ``GSS
Data Explorer \textbar{} NORC at the University of Chicago.''
\url{https://gssdataexplorer.norc.org/variables/5067/vshow}.

\leavevmode\hypertarget{ref-Rstan2018}{}%
Stan Development Team. 2018. ``RStan: The R Interface to Stan.''
\url{http://mc-stan.org/}.

\leavevmode\hypertarget{ref-Tough2017}{}%
Tough, Hannah, Johannes Siegrist, and Christine Fekete. 2017. ``Social
relationships, mental health and wellbeing in physical disability: A
systematic review.'' \emph{BMC Public Health} 17 (1). BMC Public Health:
1--18. \url{https://doi.org/10.1186/s12889-017-4308-6}.

\leavevmode\hypertarget{ref-UN2018}{}%
UN. 2018. ``Global Cost of Corruption at Least 5 Per Cent of World Gross
Domestic Product, Secretary-General Tells Security Council, Citing World
Economic Forum Data.''
\url{https://www.un.org/press/en/2018/sc13493.doc.htm}.

\leavevmode\hypertarget{ref-Vanberg2008}{}%
Vanberg, Christoph. 2008. ``Why Do People Keep Their Promises? An
Experimental Test of Two Explanations1.'' \emph{Econometrica} 76 (6):
1467--80. \url{https://doi.org/10.3982/ECTA7673}.

\leavevmode\hypertarget{ref-Vuorre2018}{}%
Vuorre, Matti, and Niall Bolger. 2018. ``Within-subject mediation
analysis for experimental data in cognitive psychology and
neuroscience.'' \emph{Behavior Research Methods} 50 (5): 2125--43.
\url{https://doi.org/10.3758/s13428-017-0980-9}.

\leavevmode\hypertarget{ref-Xu2016}{}%
Xu, Sihua, Yu Pan, You Wang, Andrea M Spaeth, Zhe Qu, and Hengyi Rao.
2016. ``Real and hypothetical monetary rewards modulate risk taking in
the brain.'' \emph{Scientific Reports} 6.
\url{https://doi.org/10.1038/srep29520}.

\bibliographystyle{spbasic}
\bibliography{bibliography.bib}

\end{document}
